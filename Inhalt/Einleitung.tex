\chapter{Einleitung}
Quellenrekonstruktion beim Fötus bietet eine Möglichkeit die Funktion
und Entwicklungsstufe des fötalen Gehirns zu untersuchen. Ein
Anwendungsgebiet ist die Untersuchung des auditorischen Kortex als Teil
des zentralen Nervernsystems und des Hörsystems und somit eines
sensorischen Apparates. Die auditorische Stimulation ist sehr geeignet
um die Reaktionen des Fötus zu überprüfen da es eine nichtinvasive,
schnelle, einfach erzeugbare und variierbare (in Frequenz und
Lautstärke) Erregung einer komplexen Nervenbahn ist. Auditorische
Stimuli erzeugen beim gesunden Fötus auditorisch evozierte Felder,
deren Quellen im auditorischen Kortex des Fötus positioniert sind. Ein
weiterer Vorteil ist, dass ein auditorischer Stimulus nicht mit
elektrischem Stromfluss und somit nicht mit störenden Magnetfeldern
verbunden ist.

Um eine Quellenrekonstruktion, basierend auf MEG-Messungen, durchführen
zu können, muss ein Modell des Volumenleiters, in dem sich die gesuchte
Quelle befindet, erstellt werden. Quellenrekonstruktion wird beim Fötus
häufig nur von gemessenen Magnetfeldern und nicht von den elektrischen
Feldern durchgeführt, da das elektrische Feld durch die Vernix caseosa
in der späteren Schwangerschaftszeit weitgehend abgeschirmt ist. Das
Magnetfeld wird in einem elektrisch abgeschirmten Raum mit einem
SQUID-Sensorsystem über dem Abdomen der Mutter aufgenommen, dabei ist
das Signal-Rausch-Verhältnis (SNR) sehr klein, was hohe Anforderungen
an die Quellenstabilität des verwendeten Modells stellt. In diesem Fall
besteht der Volumenleiter näherungsweise aus dem Fötus, der Vernix
caseosa und dem Abdomen der Mutter. Der Ansatz für die Modellierung mit
hoher Genauigkeit ist ein realistisches 3D-Modell, wobei jedem
Volumenelement eine spezifische elektrische Leitfähigkeit zugeordnet
wird. Damit lassen sich Anisotropien und Inhomogenitäten der
elektrischen Leitfähigkeit im Volumenleiter modellieren, die Werte für
die Leitfähigkeiten können beispielsweise aus MRT- und CT-Aufnahmen
geschätzt werden. Gegenüber analytischen Modellen besitzen die
realistischen Volumenleitermodelle den Vorteil, dass sie die
menschliche Anatomie wesentlich exakter beschreiben, was zu einer
verbesserten Genauigkeit der Quellenrekonstruktion führt \cite{a2}.

3D-Modelle haben den Nachteil, dass sie sehr komplex sind und
dementsprechend sind die Berechnungen mit diesen Modellen sehr
aufwendig und zeitintensiv. Die Alternative sind 2.5-D Modelle wie
\textit{Boundary-Element-Method}{}-Modelle (BEM-Modelle), die hier
verwendet werden. Dabei werden nur die Oberflächen einzelner
Volumenleiterschichten unterschiedlicher Leitfähigkeit modelliert.
Inhomogenitäten und Anisotropien der Leitfähigkeit in den Schichten
werden vernachlässigt \cite{a1}. Eine weitere Vereinfachung ist die
Diskretisierung der Oberflächen zwischen den Schichten
unterschiedlicher Leitfähigkeit mit Dreiecken (Randelementen)
\cite{a1}. Der Vorteil dieser Vereinfachungen ist die deutliche
Komplexitätsreduzierung der Modelle, damit vereinfachen sich die
Berechnungen und die Rechenzeiten verkürzen sich. Es wird ein 3-Schalen
BEM-Modell mit den Schichten Abdomen, Vernix caseosa und Fötus
verwendet. Die Rechenzeit spielt bei der Quellenrekonstruktion eine
wichtige Rolle, daher wird in dieser Arbeit eine Optimierung von
Diskretisierungsfehlern und Exaktheit der der Lösung der
Vorwärtsrechnung und des inversen Problems gesucht.

Um die Diskretisierungsfehler der Modelle vergleichen zu können wurden
Vorwärtsrechnungen mit verschiedenen Quellen durchgeführt, wobei die
entstehenden Messwerte des Magnetfeldes der Quellen in einem
Sensorsystem von jedem Modell berechnet wurden. Dafür wurden zunächst
Quellen im Gehirn des Fötus definiert. Bei der Auswertung wurden die
RDM- und MAG-Werte der Simulierten Daten jeweils mit denen des
höchstaufgelösten Modells berechnet und daraus statistische Parameter
ermittelt. Die Simulationsergebnisse des Modells mit der feinsten
Diskretisierung wurden als Referenzfelder für Quellenrekonstruktionen
mit allen Modellen verwendet. Die Auswirkungen der
Diskretisierungsabstände auf die Quellenrekonstruktion wurde durch
Winkelabweichungen und Verhältnisse der Strärken von definierten und
rekonstruierten Quellen bewertet. Vergleichen wurden jeweils die
Modelle mit gleicher Segmentierungsgrundlage und gleicher
Vernixschichtdicke also gleicher Geometrie aber unterschiedlicher
Diskretisierung, dann wurden Modelle mit gleicher Diskretisierung in
allen Schichten aber unterschiedlicher Geometrie verglichen und als
drittes wurden Modelle mit gleicher Diskretisierung und gleicher
Segmentierungsgrundlage aber verschiedener Vernixschichtdicke
verglichen.

