
\chapter{Zusammenfassung}
Die Rekonstruktion von magnetischen Quellen im Gehirn des Fötus über
eine Lösung des inversen Problems von aufgenommenen fötalen
Magnetoencephalogrammen, kann Aufschluss über die fötale Entwicklung
geben. In dieser Arbeit wird untersucht, welche realistischen
Volumenleitermodelle für eine Quellenrekonstruktion, ausgehend vom
fötalen Magnetoencephalogramm, geeignet sind. Dabei wird die
Untersuchung auf die \textit{Boundary-Element-Methode} (BEM)
eingeschränkt, wobei als Fötusmodell einmal der gesamte Fötus und
einmal nur der Kopf des Fötus dient. Ziel ist es ein Modell zu finden,
welches genaue Ergebnisse liefert und dennoch möglichst wenig
rechenintensiv ist.

Es wurden mit allen Modellen Vorwärtsrechnungen für gegebene
Quellenparameter mit identisch festgelegten Positionen, Ausrichtungen
und Amplituden der Dipole durchgeführt. Aus den Ergebnissen der
Vorwärtsrechnungen wurden die Abweichungen der simulierten Felder der
einzelnen Modelle bezogen auf ein Referenzmodell berechnet. Als
Referenzmodell wurden für alle Modelle das Fötusmodell mit dem feinsten
Gitter und zusätzlich für die Kopfmodelle das Kopfmodell mit dem
feinsten Gitter verwendet. Als Größen für die Bewertung der
Abweichungen wurden die MAG- und RDM-Werte berechnet. Im nächsten
Schritt wurden die Modelle auf ihre Genauigkeit in bezug auf das
inverse Problem untersucht. Als Datensatz für die Quellenrekonstruktion
wurde bei allen Modellen das Simulationsergebnis des höchstaufgelösten
Fötusmodell verwendet.

Es hat sich herausgestellt, dass die Modelle in denen der gesamte Fötus
modelliert wurde, mit einer verhältnismäßig groben Diskretisierung für
eine Quellenrekonstruktion der optimale Kompromiss zwischen Rechenzeit
und Quellenstabilität ist.
