\chapter{Zusammenfassung und Ausblick}
Die Rekonstruktion von magnetischen Quellen im Gehirn des F�tus �ber
eine L�sung des inversen Problems von aufgenommenen f�talen
Magnetoencephalogrammen, kann Aufschluss �ber die f�tale Entwicklung
geben. In dieser Arbeit wird untersucht, welche realistischen
Volumenleitermodelle f�r eine Quellenrekonstruktion, ausgehend vom
f�talen Magnetoencephalogramm, geeignet sind. Dabei wird die
Untersuchung auf die \textit{Boundary-Element-Methode} (BEM)
eingeschr�nkt, wobei als F�tusmodell einmal der gesamte F�tus und
einmal nur der Kopf des F�tus dient. Ziel ist es ein Modell zu finden,
welches genaue Ergebnisse liefert und dennoch m�glichst wenig
rechenintensiv ist.

Es wurden mit allen Modellen Vorw�rtsrechnungen f�r gegebene
Quellenparameter mit identisch festgelegten Positionen, Ausrichtungen
und Amplituden von Dipolen durchgef�hrt. Aus den Ergebnissen der
Vorw�rtsrechnungen wurden die Abweichungen der simulierten Felder der
einzelnen Modelle bezogen auf ein Referenzmodell berechnet. Als Gr��en
f�r die Bewertung der Abweichungen wurden Amplitudenabweichungen und
\textit{relative differece measures} (als Ma� f�r die
Topologieabweichung) berechnet. Im n�chsten Schritt wurden die Modelle
auf ihre Genauigkeit in Bezug auf das inverse Problem untersucht. Als
Datensatz f�r die Quellenrekonstruktion wurde bei allen Modellen das
Simulationsergebnis eines geeigneten Referenzmodells mit hoher
Aufl�sung verwendet und es wurden die Amplituden- und
Orientierungsfehler der rekonstruierten Dipole berechnet.

Es hat sich herausgestellt, dass die Modelle in denen der gesamte F�tus
modelliert wurde, mit einer Diskretisierung durch eine Dreiecksseitenl�nge 
von 5mm in den inneren Schichten, f�r
eine Quellenrekonstruktion der optimale Kompromiss zwischen Rechenzeit
und Quellenstabilit�t ist. Eine Ver�nderung der Schichtdicke der Vernix
caseosa in den BEM-Modellen hatte nur einen sehr geringen Einfluss auf
die durchgef�hrten Vorw�rtsrechnungen und Quellenrekonstruktionen.
Die Vernachl�ssigbarkeit von kleinen �nderungen der Schichtdicke der Vernix caseosa
muss f�r den allgemeinen Fall noch validiert werden, da hier unter idealisierten 
Bedingungen gearbeitet wurde. 

Eine Validierung der Ergebnisse sollte f�r das direkte und inverse Problem verschiedene Modelle
einsetzen, \dh beispielsweise FEM-Modelle f�r die Vorw�rtsrechnung und BEM-Modelle f�r die L�sung
des inversen Problems, um die Ergebnisse der L�sung des inversen Problems verl�sslich auswerten zu k�nnen.
Dabei k�nnte auch untersucht werden wie sich das Fehlen der Vernix caseosa in BEM-Modellen auf die L�sung des inversen Problems auswirkt.