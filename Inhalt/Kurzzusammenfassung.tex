\chapter{Kurzzusammenfassung}
F�r klinische Fragestellungen ist die Exaktheit der r�umlichen Lokalisierung von elektrischer Aktivit�t im f�talen Gehirn von gro�em Interesse. Eine g�ngige Methode zur Registrierung der f�talen Hirnaktivit�t ist die Magnetoenzephalographie (MEG). Die Quellenrekonstruktion in der MEG ben�tigt eine Volumenleitermodellierung, welche zunehmend auf der Randelementemethode (boundary element method, BEM) basiert \cite{Haueisen1997}. In dieser Arbeit wird die Genauigkeit der L�sung der Vorw�rtsrechnung und des inversen Problems in Abh�ngigkeit von der Randelementediskretisierung bei 3-Schalen BEM-Modellen in der MEG quantifiziert. Die inneren Schalen wurden  auf zwei verschiedene Arten modelliert, wobei einmal die Segmentierung des F�tuskopfes und zum anderen die Segmentierung des gesamten F�tus verwendet wurde. Des weiteren wurde die Schichtdicke der Vernix caseosa variiert. Zur Pr�fung der Modelle wurden Simulationen mit Einzeldipolen durchgef�hrt und die Simulationsergebnisse eines gew�hlten Referenzmodells wurden f�r die Rekonstruktion der Einzeldipole mit jedem Volumenleitermodell verwendet. Es zeigte sich, dass die Variation der Schichtdicke der Vernix caseosa um 1mm bis 2mm in diesem Szenario einen vernachl�ssigbaren Einfluss auf die L�sung des direkten und inversen Problems hat. Der Unterschied zwischen den zwei Segmentierungsgrundlagen zeigte sich speziell bei der r�umlichen Rekonstruktion der Dipole im inversen Problem, dabei entstanden Fehler in der rekonstruierten Dipolorientierung von �ber 60�. Auch die Randelementediskretisierung hat einen deutlichen Einfluss auf die L�sung der Vorw�rtsrechnung und des inversen Problems, dabei haben Dreiecksseitenl�ngen von �ber 5mm in den inneren Schichten erhebliche Abweichungen verursacht. Unter Ber�cksichtigung der Knotenanzahl \bzw der Anzahl der Randelemente, erzielt das Modell, dessen innerste Schale den gesamten F�tus modelliert, mit einer Dreiecksseitenl�nge von 5mm in den inneren Schichten, gute Ergebnisse.